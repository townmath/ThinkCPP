% LaTeX source for textbook ``Think C++''
% Copyright (C) 1999  Allen B. Downey


\documentclass{book}
\usepackage{epsfig}
\usepackage{makeidx}
\usepackage{url}
\usepackage{color}
\definecolor{dkgreen}{rgb}{0,0.6,0}
\definecolor{gray}{rgb}{0.5,0.5,0.5}
\definecolor{mauve}{rgb}{0.58,0,0.82}
\usepackage{listings}
\lstset{frame=tb,
   language=C++,
   aboveskip=3mm,
   belowskip=3mm,
   showstringspaces=false,
   columns=flexible,
   basicstyle={\small\ttfamily},
   numbers=none,%left,
   numberstyle=\tiny\color{gray},
   keywordstyle=\color{blue},
   commentstyle=\color{dkgreen},
   stringstyle=\color{mauve},
   breaklines=true,
   frame=none,
   breakatwhitespace=true,
   tabsize=3
   }
\usepackage{hyperref}
\hypersetup{
    colorlinks,
    citecolor=black,
    filecolor=black,
    linkcolor=black,
    urlcolor=black
}
\pssilent

\title{Think C++}

\author{Allen B. Downey}
\date{}

\sloppy
\setlength{\topmargin}{0.75in}
\setlength{\headsep}{0.5in}
\setlength{\oddsidemargin}{1.0in}
\setlength{\evensidemargin}{.95in}
\makeindex

\begin{document}

\title {Think C++}
\author {Allen B. Downey \\ adapted for CISP 360 by \\ James R. Town}
\date {Version 1.1.361}
\maketitle

\vspace{2in}
\begin{center}
{\Large Think C++}

\vspace{0.25in}

Copyright (C) 1999  Allen B. Downey
\end{center}
\vspace{0.25in}

\begin{flushleft}
Green Tea Press       \\
9 Washburn Ave \\
Needham MA 02492
\end{flushleft}

Permission is granted to copy, distribute, transmit and adapt this
work under the Creative Commons Attribution-NonCommercial-ShareAlike 4.0
International License: \url{https://creativecommons.org/licenses/by-nc/4.0/}.

If you are interested in distributing a commercial version of this
work, please contact the author.

The \LaTeX\ source and code for this edition of the book is available from

\begin{verbatim}
https://github.com/townmath/ThinkCPP
\end{verbatim}

The \LaTeX\ source and code for the original book is available from

\begin{verbatim}
https://github.com/AllenDowney/ThinkCPP
\end{verbatim}

\frontmatter
\tableofcontents

\mainmatter
\include{cha1}
\include{cha2}
\include{cha3}
\include{cha4}
\include{cha5}
\include{cha6}
\include{cha7}
\include{cha8}
% LaTeX source for textbook ``How to think like a computer scientist''
% Copyright (C) 1999  Allen B. Downey


\chapter{File Input/Output and 2D Arrays}

In this chapter we will develop a program that reads and writes files,
parses input, and demonstrates 2D Arrays.  We will also
implement a data structure called {\tt Set} that expands automatically
as you add elements.

Aside from demonstrating all these features, the real purpose of the
program is to generate a two-dimensional table of
the distances between cities in the United States.
The output is a table that looks like this:

\begin{verbatim}
Atlanta 0
Chicago 700     0
Boston  1100    1000    0
Dallas  800     900     1750    0
Denver  1450    1000    2000    800     0
Detroit 750     300     800     1150    1300    0
Orlando 400     1150    1300    1100    1900    1200    0
Phoenix 1850    1750    2650    1000    800     2000    2100    0
Seattle 2650    2000    3000    2150    1350    2300    3100    1450    0
        Atlanta Chicago Boston  Dallas  Denver  Detroit Orlando Phoenix Seattle
\end{verbatim}
%
The diagonal elements are all zero because that is the distance
from a city to itself.  Also, because
the distance from A to B is the same as the distance from B
to A, there is no need to print the top half of the matrix.

\section {Streams}
\index{stream}

To get input from a file or send output to a file, you have to
create an {\tt ifstream} object (for input files) or an
{\tt ofstream} object (for output files).  These objects
are defined in the header file {\tt fstream}, which you
have to include.

\index{header file}

A {\bf stream} is an abstract object that represents the flow
of data from a source like the keyboard or a file to a destination
like the screen or a file.

We have already worked with two streams: {\tt cin}, which has type
{\tt istream}, and {\tt cout}, which has type {\tt ostream}.
{\tt cin} represents the flow of data from the keyboard to
the program.  Each time the program uses the {\tt >>} operator
or the {\tt getline} function, it removes a piece of data
from the input stream.

\index{cin}
\index{cout}
\index{istream}
\index{ostream}

Similarly, when the program uses the {\tt <<} operator on
an {\tt ostream}, it adds a datum to the outgoing stream.

\section {File input}
\label{finput}
\index{file!input}
\index{ifstream}

To get data from a file, we have to create a stream that flows
from the file into the program.  
We can do that using the {\tt ifstream} constructor.

\begin{lstlisting}
  ifstream infile ("file-name");
\end{lstlisting}
%
The argument for this constructor is a string that
contains the name of the file you want to open.  The result
is an object named {\tt infile} that supports all the same
operations as {\tt cin}, including {\tt >>} and {\tt getline}.

\begin{lstlisting}
  int x;
  string line;
    
  infile >> x;               // get a single integer and store in x
  getline (infile, line);    // get a whole line and store in line
\end{lstlisting}
%
If we know ahead of time how much data is in a file, it is 
straightforward to write a loop that reads the entire file and
then stops.  More often, though, we want to read the entire
file, but don't know how big it is.

There are member functions for {\tt ifstreams} that check the status
of the input stream; they are called {\tt good}, {\tt eof}, {\tt fail}
and {\tt bad}.  We will use {\tt good} to make sure the file was
opened successfully and {\tt eof} to detect the ``end of file.''

\index{stream!status}
\index{good}
\index{eof}
\index{end of file}

Whenever you get data from an input stream, you don't
know whether the attempt succeeded until you check.  If the
return value from {\tt eof} is {\tt true} then we have reached
the end of the file and we know that the last attempt failed.
Here is a program that reads lines from a file and displays
them on the screen:

\begin{lstlisting}
  string fileName = ...;
  ifstream infile (fileName);

  if (infile.good() == false) {
    cout << "Unable to open the file named " << fileName;
    exit (1);
  }

  while (true) {
    getline (infile, line);
    if (infile.eof()) break;
    cout << line << endl;
  }
\end{lstlisting}
%

Immediately after opening the file, we invoke the {\tt good} function.
The return value is {\tt false} if the system could not open the file,
most likely because it does not exist, or you do not have permission
to read it.

\index{loop!infinite}
\index{infinite loop}

The statement {\tt while(true)} is an idiom for an infinite
loop.  Usually there will be a {\tt break} statement somewhere in
the loop so that the program does not really run forever (although
some programs do).  In this case, the {\tt break} statement allows
us to exit the loop as soon as we detect the end of file.

\index{break statement}
\index{statement!break}
\index{getline}

It is important to exit the loop between the input statement and
the output statement, so that when {\tt getline} fails at the
end of the file, we do not output the invalid data in {\tt line}.

\section{File output}
\index{file output}
\index{ofstream}

Sending output to a file is similar.  For example, we could
modify the previous program to copy lines from one file to
another.

\begin{lstlisting}
  ifstream infile ("input-file");
  ofstream outfile ("output-file");

  if (infile.good() == false || outfile.good() == false) {
    cout << "Unable to open one of the files." << endl;
    exit (1);
  }

  while (true) {
    getline (infile, line);
    if (infile.eof()) break;
    outfile << line << endl;
  }
\end{lstlisting}

\section{Parsing input}
\label{parsing}
\index{parsing}

In Section~\ref{formal} I defined ``parsing'' as the process of
analyzing the structure of a sentence in a natural language or a
statement in a formal language.  For example, the compiler has to
parse your program before it can translate it into machine language.

In addition, when you read input from a file or from the keyboard
you often have to parse it in order to extract the information
you want and detect errors.

For example, I have a file called {\tt distances} that contains
information about the distances between major cities in the
United States.  I got this information from a randomly-chosen
web page

\begin{lstlisting}
http://www.jaring.my/usiskl/usa/distance.html
\end{lstlisting}
%
so it may be wildly inaccurate, but that doesn't matter.  The
format of the file looks like this:

\begin{verbatim}
"Atlanta"       "Chicago"       700
"Atlanta"       "Boston"        1,100
"Atlanta"       "Chicago"       700
"Atlanta"       "Dallas"        800
"Atlanta"       "Denver"        1,450
"Atlanta"       "Detroit"       750
"Atlanta"       "Orlando"       400
\end{verbatim}
%
Each line of the file contains the names of two cities in quotation
marks and the distance between them in miles.  The quotation marks
are useful because they make it easy to deal with names that have
more than one word, like ``San Francisco.''

By searching for the quotation marks in a line of input, we
can find the beginning and end of each city name.
Searching for special characters like quotation marks can be a little
awkward, though, because the quotation mark is a special character
in C++, used to identify string values.

If we want to find the
first appearance of a quotation mark, we have to write something
like:

\begin{lstlisting}
  int index = line.find ('\"');
\end{lstlisting}
%
The argument here looks like a mess, but it represents a single
character, a double quotation mark.  The outermost single-quotes
indicate that this is a character value, as usual.  The backslash
(\verb+\+) indicates that we want to treat the next character
literally.  The sequence \verb+\"+ represents a quotation mark; the
sequence \verb+\'+ represents a single-quote.  Interestingly, the
sequence \verb+\\+ represents a single backslash.  The first backslash
indicates that we should take the second backslash seriously.

\index{special character}
\index{character!special sequence}
\index{backslash}

Parsing input lines consists of finding the beginning and
end of each city name and using
the {\tt substr} function to extract the cities and distance.
{\tt substr} is a {\tt string} member function;
it takes two arguments, the starting index of the substring
and the length.

\index{find}

\begin{lstlisting}
void processLine (string line, string& city1, string& city2, string& distString) {
  // the character we are looking for is a quotation mark
  char quote = '\"';

  // store the indices of the quotation marks in an array
  int quoteIndex[4];

  // find the first quotation mark using the built-in find
  quoteIndex[0] = line.find (quote);

  // find the other quotation marks using the find from Chapter 7
  for (int i=1; i<4; i++) {
    quoteIndex[i] = find (line, quote, quoteIndex[i-1]+1);
  }

  // break the line up into substrings
  int len1 = quoteIndex[1] - quoteIndex[0] - 1;
  city1 = line.substr (quoteIndex[0]+1, len1);
  int len2 = quoteIndex[3] - quoteIndex[2] - 1;
  city2 = line.substr (quoteIndex[2]+1, len2);
  int len3 = line.length() - quoteIndex[2] - 1;
  distString = line.substr (quoteIndex[3]+1, len3);

  // output the extracted information
  cout << city1 << "\t" << city2 << "\t" << distString << endl;
}
\end{lstlisting}
%
Of course, just displaying the extracted information is not
exactly what we want, but it is a good starting place.

\section{Parsing numbers}
\label{parsingnums}
\index{parsing number}
\index{atoi}
\index{convert!to integer}

The next task is to convert the numbers in the file from strings to
integers.  When people write large numbers, they often use commas to
group the digits, as in 1,750.  Most of the time when computers write
large numbers, they don't include commas, and the built-in functions
for reading numbers usually can't handle them.  That makes the
conversion a little more difficult, but it also provides an
opportunity to write a comma-stripping function, so that's ok.  Once
we get rid of the commas, we can use the library function {\tt atoi}
to convert to integer.  {\tt atoi} is defined in the header file {\tt
cstdlib}.

\index{character!classification}
\index{isdigit}

To get rid of the commas, one option is to traverse the string and
check whether each character is a digit.  If so, we add it to the
result string.  At the end of the loop, the result string contains all
the digits from the original string, in order.

\begin{lstlisting}
int convertToInt (string s)
{
  string digitString = "";

  for (int i=0; i<s.length(); i++) {
    if (isdigit (s[i])) {
      digitString += s[i];
    }
  }
  return atoi (digitString.c_str());
}
\end{lstlisting}
%
The variable {\tt digitString} is an example of an {\bf accumulator}.  It is
similar to the counter we saw in Section~\ref{loopcount},
except that instead of getting incremented, it gets accumulates
one new character at a time, using string concatentation.

The expression

\begin{lstlisting}
      digitString += s[i];
\end{lstlisting}
%
is equivalent to

\begin{lstlisting}
      digitString = digitString + s[i];
\end{lstlisting}
%
Both statements add a single character onto the end of the existing
string.

\index{concatentation}
\index{string!concatentation}
\index{accumulator}
\index{pattern!accumulator}

Since {\tt atoi} takes a C string as a parameter, we have
to convert {\tt digitString} to a C string before passing it
as an argument.

\index{c\_str}
\index{C string}
\index{string!native C}

\section {{\tt matrix}}
\index{matrix}
\index{matrix}

A {\tt matrix} is a two-dimensional array.  Instead of a length, it has two
dimensions, rows and columns.

Each element in the matrix is identified by two indices;
one specifies the row number, the other the column number.

\index{index}

Create a matrix similarly to an array:

\begin{lstlisting}
  const int MAXROWS=50;
  const int MAXCOLS=50;
  int matrix[MAXROWS][MAXCOLS];
\end{lstlisting}
%
This creates an array of integers with MAXROWS rows and MAXCOLS columns.  You can create an array of any of the datatypes we have learned about.  

To access the elements of a matrix, we use the {\tt []} operator
to specify the row and column:

\begin{lstlisting}
  matrix[0][0] = 1;
  matrix[1][2] = 10.0 * m2[0][0];
\end{lstlisting}
%
If we try to access an element that is out of range, the program
prints an error message and quits.

\index{run-time error}

Remember that the row indices run from 0 to
{\tt MAXROWS -1} and the column indices run from 0 to
{\tt MAXCOLS -1}.

\index{loop!nested}

The usual way to traverse a matrix is with a nested loop.
This loop sets each element of the matrix to the sum of its
two indices:

\begin{lstlisting}
  for (int row=0; row < MAXROWS; row++) {
    for (int col=0; col < MAXCOLS; col++) {
      matrix[row][col] = row + col;
    }
  }
\end{lstlisting}
%
This loop prints each row of the matrix with tabs between the
elements and newlines between the rows:

\begin{lstlisting}
  for (int row=0; row < MAXROWS; row++) {
    for (int col=0; col < MAXCOLS; col++) {
      cout << matrix[row][col] << "\t";
    }
    cout << endl;
  }
\end{lstlisting}
%

\section{A distance matrix}

Finally, we are ready to put the data from the file into
a matrix.  Specifically, the matrix will have one row and
one column for each city.

We'll create the matrix in {\tt main}, with plenty of space
to spare:

\begin{lstlisting}
  int distances [MAXROWS][MAXCOLS];
\end{lstlisting}
%

Next we add new information to the
matrix by using the indices of the two cities:

\begin{lstlisting}
void addDataToMatrix(string city1, string city2, string distString, string cityArr[],int &numCities, int distances[][MAXCOLS]){
    //this find is adapted from the chapter 7 version
    int index1 = find(cityArr,city1,numCities);
    if (index1<0){ //city not found
        index1=numCities;
        cityArr[numCities++]=city1;
    }
    int index2=find(cityArr,city2,numCities);
    if (index2<0){ //city not found
        index2=numCities;
        cityArr[numCities++]=city2;
    }
    int dist = convertToInt (distString);
    distances[index1][index2] = dist;
    distances[index2][index1] = dist;
}
\end{lstlisting}
%
Finally, in {\tt main} we can print the information in a
human-readable form:

\begin{lstlisting}
for (int i=0; i<numCities; i++) {
     for (int j=-1; j<=i; j++) {
         if (j<0){
             cout<<cityArr[i]<<"\t";
         } else {
             cout << distances[i][j] << "\t";
         }
    }
    cout << endl;
}
cout<<"\t";
for (int i=0; i<numCities; i++){
    cout<<cityArr[i]<<"\t";
}
cout << endl;
\end{lstlisting}
%
This code produces the output shown at the beginning of the
chapter.  The original data is available from this book's github page.

\section{A proper distance matrix}

Although this code works, it is not as well organized as it
could be, once we learn about vectors and objects, we might be able to improve upon this early prototype.  

\section{Another type of Stream}
\index{stringstream}
\index{stringstream}

A {\tt stringstream} is another way to move data in your program.  Like the other streams we have learned, it works with both the {\tt >>} operator
and the {\tt getline} function.  You will often use this in conjunction with an {\tt ifstream} to help parse your data.  Let's say it looks like the following table where each line is a different person's data, but you don't know how much data is on each line. 
\index{stringstream}

\begin{verbatim}
hgt:170 eye:blue age:23 
hgt:165 age:45
hgt:200 eye:amber age:31 wgt:200
\end{verbatim}

You could read and print the data for each person like so:
\lstinputlisting{../code/cha9/cha9.cpp}

As you can see, streams can be very useful. 

\section{Glossary}

\begin{description}

\item[ordered set:]  A data structure in which every element appears
only once and every element has an index that identifies it.

\item[stream:]  A data structure that represents a ``flow'' or
sequence of data items from one place to another.  In C++ streams
are used for input and output.

\item[accumulator:]  A variable used inside a loop to accumulate
a result, often by getting something added or concatenated during each
iteration.

\index{ordered set}
\index{set!ordered}
\index{stream}
\index{accumulator}
\index{pattern!accumulator}

\end{description}



\include{cha10}
\include{cha11}
\include{cha12}
\include{cha13}
\include{cha14}
\include{cha15}

\printindex

\end{document}

